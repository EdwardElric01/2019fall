\documentclass[UTF8]{ctexart}
\usepackage{amsmath, amssymb}
\title{统计推断 第二周}
\author{Edward}
\begin{document}
\maketitle

\section{Muller Methods}%
\label{sec:muller_methods}

如何生成一个二维标准正态随机变量?

生成下面的随机数 \[
    \begin{cases}
        \frac{r^{2}}{2} \sim Exp \left( 1 \right) \\
        \theta \sim U\left(0,1  \right) 
    \end{cases}
.\] 


然后进行坐标变换 \[
    \begin{cases}
        x &= r \cos \theta \\
        y &= r \sin \theta \\
    \end{cases}
.\] 


\section{取舍原则}%
\label{sec:取舍原则}

生成密度为 $ p_0\left(x  \right)  $ 


 \[
     Pr\left(\ita<x  \right)  &=  Pr\left( \ksai <x | \kai _k kept \right)  \\
.\] 


\section{MCMC}%
\label{sec:mcmc}

Markov Chain的遍历性定理: 若有限状态 Markov chain 的所有状态都是互通的,f 是状态空间上的youjie实值函数且满足 $ \sum_{i} $ 则

\[
    Pr\left( \frac{f\left(\ksai_1  \right) + \dots + f\left( \ksai_n  \right)  }{n} \to \sum_{i} f\left(\ksai_i  \right) \pi_i \right) = 1 
.\] 

其中 $ \pi $  是Markov chain 的不变分布,而且此极限与初始分布无关

优点:动态地生成随机数。可以用来生成高维的概率分布


\end{document}

\documentclass[UTF8]{ctexart}
\usepackage{minted} 
\usepackage{indentfirst} 
\usepackage{amsmath, amssymb}
\usepackage{geometry}
\geometry{a4paper, scale=0.8}

\begin{document}
\title{作业:引论}
\maketitle
\section{拿石子问题}%
\label{sec:拿石子问题}

\begin{enumerate}
    \item \textbf{证明} n为Fibonacci数时乙有必胜策略

\textbf{proof} \textbf{数学归纳法} 

设 Fibonacci 序列为 $\left\{ F_n \right\} _{n=2} ^{\infty}, F_2 = 2$ 
\begin{enumerate}
    \item $n=2$时,可能的取法只有一种,甲取1个,乙取1个,显然乙必胜。

    \item 假设 $k\le n$ 时乙必胜,当 $k=n+1$ 时,由Fibonacci数的性质, \[
    F _{n+1} = F_n + F _{n-1}
    .\] 

    \begin{enumerate}
        \item 若 $m _{\text{甲}}^{1} \ge F _{n-1}$,由于 $F _{n+1} = F_n + F _{n-1} < 3 F_n$,$m _{\text{乙}} ^{1} = F _{n+1} - m _{\text{甲}} ^{1} < 2 F _{n-1} < 2 m _{\text{甲}} ^{1}$,故此时乙可胜利
        \item 若 $m _{\text{甲}}^{1} < F _{n-1}$,则可以将石子看做两堆,一堆 $F_n$ 个(记作A),一堆  $F _{n-1}$ 个(记作B),由于 $m _{\text{甲}} ^{1} < F _{n-1}$,因此可以看作甲、乙两人先从B堆中开始拿。由归纳假设,此时乙有策略使乙恰好拿完B的最后一颗石子。


            下面说明在乙取完B堆的最后一颗石子之后,甲下次不可能取完A堆。设乙第 $ l $ 次时正好取完了B堆的最后一颗石子,则易知 $m _{\text{乙}} ^{1} \le \frac{2}{3} F_{n-1}$,因此只需要比较 $ \frac{4}{3} F_{n-1} $ 和 $ F_{n} $ 的大小即可, 相当于比较 $ 4 F_{n-1} $ 和 $ 3 F_{n} $ 的大小 \[
                3 F_{n} - 4 F_{n-1} =  3 F_{n-2} - F_{n-1} > 0
            .\] 

        即 $ F_n > \frac{4}{3} F_{n-1}$ ,因此,下次甲最多拿 $ \frac{4}{3} F_{n-1} $ 个石子,不会取完A堆。


        这说明,在乙取完B堆后,问题转化为甲乙两从石子数为 $ F_{n} $ 重新开始游戏,由数学归纳法原理,乙有必胜策略。

        \end{enumerate}
    \end{enumerate}


\item \textbf{证明} n不是 Fibonacci数时,甲有必胜策略。
    不会
\end{enumerate}

\section{求递归式的渐近表示}%
\label{sec:求递归式的渐近表示}

\begin{enumerate}
    \item $ T\left( n \right) = T \left( n-1 \right) + n  $ 

        \textbf{solution:}
        \begin{align*}
            T\left( n \right)  - T\left( n-1 \right)  = n  \\
            T\left( n-1 \right)  - T\left( n-2 \right)   = n-1  \\
            \ldots \\
            T\left( 2 \right)  - T\left( 1 \right)   = 2  \\
        .\end{align*}
        将上式累加,有
        \[
            T\left( n \right) = T\left( 1 \right) - 1 +  \frac{\left( n+1 \right)n  }{2} 
        .\] 
    因此,$ T\left( n \right) = \Theta \left( n^2 \right)  $

    \item   
        \[
            \begin{cases}
                T\left( 1 \right) = T\left( 2 \right) = 1   \\
                T\left( n \right)  = T\left( n-1 \right) + T\left( n-2 \right) , n \ge 3
            \end{cases}
        .\] 

        \textbf{solution:}

        上述表达的特征方程为  \[
        x^{2} = x + 1
        .\] 
    方程的解为 \[
           x_{1,2} =  \frac{1 \pm \sqrt{2} }{2} 
    .\] 

因此,$ T\left( n \right)  $ 的表达式为 \[
    T\left( n \right)  = a \left(  \frac{1+\sqrt{2} }{2}\right) ^{n} + b \left(  \frac{1-\sqrt{2} }{2} \right)^{n}
.\] 

所以,$ T\left( n \right)  $ 的渐近表达式为  $ T\left( n \right)  = \Theta\left( 2^{n} \right) $

    \item $ T\left( n \right)  = T\left( \left| \sqrt{n}  \right|  \right) + 1 $

        \textbf{solution:}

        设 $ n = 2^{k} $ 则 $ k = \log_{2} n $ ,有 

        \begin{align*}
            T\left( n \right) &= T\left( \sqrt{2^{k}}  \right)  + 1  \\
            &= T\left( 2^{k-1}  \right) + 1 \\
            &= T\left( 2^{k-2}  \right) + 2 \\
            \ldots \\
            &= T\left( 2^{0}  \right) + k \\
            &= T \left( 1\right)  + k \\
            &= \log_{2} n
        .\end{align*}
        因此,$ T\left(n  \right)  = \Theta\left( \log n \right)  $
    \item $ T\left( n \right) = \frac{n}{n-1} T\left( n-1 \right)  + 1 $

        \textbf{solution:}
            \begin{align*}
                    T\left( n \right)  &=  \frac{n}{n-1} T \left( n-1 \right)  + 1 \\
                &= \frac{n}{n-2} T\left( n-2 \right) + \frac{n}{n-1} + 1 \\
                &= \ldots \\
                &= \frac{n}{1} T\left(1  \right)  + \frac{n}{2} + \ldots + \frac{n}{n-1} + 1 \\
                &=  n T\left( 1 \right)  + \sum_{i=2}^{n} \frac{1}{i-1} \\
                &= n T\left( 1 \right)  + n \ln n  + \epsilon_n - 1
            .\end{align*}

    其中 $ \lim_{n \to \infty} \epsilon_n = \epsilon  $  ,$ \epsilon $ 为欧拉常数 

    因此 $ T\left( n \right)  = \Theta\left(  n \log n\right) $
\end{enumerate}

\section{有序拆分问题}%
\label{sec:有序拆分问题}

下面是用python语言实现的算法:

    \begin{minted}{Python}
    import copy

    def OrderedSplit(n):
        if n == 1:
            return [[1]]
        else:
            result = [[n]]
            for element in OrderedSplit(n-1):
                result.append([1] + element)
                result.append(element + [1])
            # 去重
            new_result = copy.deepcopy(result)
            for i in range(len(result)):
                for j in range(i+1, len(result)):
                    if result[i] == result[j]:
                        new_result.remove(result[j])
            return new_result
    \end{minted}

分析:这个算法中使用到了递归,$ n+1 $ 时的输出是通过 $ n $ 时得到的。并且还包含了去重操作,设其时间复杂度为 $ T\left( n \right) $,则有下面的式子成立
\[
    \begin{cases}
        T\left( 1  \right)  = 1 \\
        T\left( n \right)  = T\left( n-1  \right)  + \Theta\left( n^{2} \right)  
    \end{cases}
.\] 

其中 $ \Theta(n^{2}) $ 是去重操作的时间复杂度。有

\[
    \begin{aligned}
        T\left( n \right) &= T\left( n-1 \right)  + \Theta\left( n^{2} \right)  \\
        &=  T\left( n-1 \right)  + c n^{2}\\ 
        &= \cdots \\
        &= T\left( 1 \right)  + c \sum_{i=2}^{n} i^{2} \\
        &=  c \sum_{i=1}^{n}i^{2} \\
        &= c \frac{1}{6} n \left( n+1 \right) \left( 2n+1 \right)  \\
        &= \Theta\left( n^{3} \right)  \\
    \end{aligned}
.\] 

\end{document}

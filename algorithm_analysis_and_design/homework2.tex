\documentclass[UTF8]{ctexart}
\usepackage{amsmath, amssymb}
\usepackage{algorithm}
\usepackage{algpseudocode} % 如果用 algorithmicx的话要用这个, algorithmic 不用
\usepackage{algorithmicx}
\usepackage{lipsum}

\makeatletter
\newenvironment{breakablealgorithm}
  {% \begin{breakablealgorithm}
   \begin{center}
     \refstepcounter{algorithm}% New algorithm
     \hrule height.8pt depth0pt \kern2pt% \@fs@pre for \@fs@ruled
     \renewcommand{\caption}[2][\relax]{% Make a new \caption
       {\raggedright\textbf{\ALG@name~\thealgorithm} ##2\par}%
       \ifx\relax##1\relax % #1 is \relax
         \addcontentsline{loa}{algorithm}{\protect\numberline{\thealgorithm}##2}%
       \else % #1 is not \relax
         \addcontentsline{loa}{algorithm}{\protect\numberline{\thealgorithm}##1}%
       \fi
       \kern2pt\hrule\kern2pt
     }
  }{% \end{breakablealgorithm}
     \kern2pt\hrule\relax% \@fs@post for \@fs@ruled
   \end{center}
  }
\makeatother

\title{分治法作业}
\author{Edward}

\begin{document}
\maketitle


\section{section name}%
\label{sec:section_name}



\section{求渐近表示}%
\label{sec:求渐近表示}


\begin{enumerate}
    \item $ T\left( n \right)  = T\left( \frac{9n}{10}\right)+n  $  

        $ a=1, b=\frac{10}{9}, \log_b a = 0 $ ,需要逐项递推,设 $ n = \frac{10}{9}^{k}$,则 $ k = \log_{\frac{10}{9}}n $  

        \[
            \begin{aligned}
                T\left( n \right) &= T\left( \frac{9n}{10} \right)  + n\\
                &=  T\left( \frac{10}{9}^{k-1} \right) + \frac{10}{9}^{k}\\
                &=  T\left( \frac{10}{9}^{k-2} \right) + \frac{10}{9}^{k-1} + \frac{10}{9}^{k}\\
                &= \cdots \\
                &= T\left( 1 \right)  + \sum_{i=1}^{k} \frac{10}{9}^{i}\\
                &= T\left( 1 \right) + \frac{\frac{10}{9} \left( 1-\frac{10}{9}^{k} \right) }{1-\frac{10}{9}}  \\
                &= T\left( 1 \right) +  10\left( \frac{10}{9}^{k}-1 \right) \\
                &= T\left( 1 \right) + 10\left( n-1 \right)  \\
                &= \Theta \left( n \right)  \\
            \end{aligned}
        .\] 

    \item $ T\left( n \right) = 2 T\left( \frac{n}{2}\right) + n^{3}  $ 

        $ a = 2, b = 2, \log_b a = 1 $ , 由 master 定理知,$ T\left( n \right) = \Theta\left( n^{3}  \right)  $ 

    \item $ T\left( n \right)  = 16 T\left( \frac{n}{4} \right) + n^{2} $  

        $ a = 16, b = 4, \log_b a = 2 $ , 由 master 定理知,$ T\left( n \right) = \Theta\left( n^{2}\log n \right)  $ 

    \item $ T\left( n \right) = 7 T\left( \frac{n}{3} \right) + n^{2} $ 

        $ a = 7, b = 3, \log_b a < 2 $ , 由 master 定理知,$ T\left( n \right) = \Theta\left( n^{2} \right)  $ 

    \item $ T\left(n  \right)  = 7 T\left( \frac{n}{2} \right) + n^{2}$ 

        $ a = 7, b = 2, \log_b a > 2 $ , 由 master 定理知,$ T\left( n \right) = \Theta\left( n^{\log_2 7} \right)  $ 

    \item $ T\left( n \right) = 2 T\left( \frac{n}{4} \right) + \sqrt{n}   $ 

        $ a = 2, b = 4, \log_b a = \frac{1}{2} $ , 由 master 定理知,$ T\left( n \right) = \Theta\left( \sqrt{n} \log n  \right)  $ 

    \item $ T\left( n \right) = T\left( \frac{2n}{3}\right) + T\left( \frac{n}{3} \right) + 2n  $ 

        $ a^{1}=1, a^{2}=1, b^{1}=\frac{3}{2}, b^{2} = 3 $, 则满足 $(\frac{2}{3})^{p} + (\frac{1}{3})^{p} =1  $ 的 p 有唯一值,且为 1。因此,由 master 定理的推广形式,$ T\left( n \right) = \Theta\left( n \log n \right)  $ 



\end{enumerate}


\section{设计算法寻找单峰分布的峰值}%
\label{sec:设计算法寻找单峰分布的峰值}


\textbf{solution}:

    算法如下:

\renewcommand{\algorithmicrequire}{ \textbf{Input:}}
\renewcommand{\algorithmicensure}{ \textbf{Output:}}

\begin{breakablealgorithm}
\caption{寻找单峰序列的峰值}
\begin{algorithmic}[1] %这个1 表示每一行都显示数字
\Procedure{FindPeak}{A(1:m)}
\If{$ m==1 $ }
    \State \Return A(1)
\EndIf

\If{$ m==2 $ }
    \If{$ A\left( 1 \right) > A\left( 2 \right)  $ }
        \State \Return A(1)
    \Else
        \State \Return A(2)
    \EndIf
\EndIf

\If{$ m==3 $ }
    \If{$ A\left( 1 \right) > A\left( 2 \right) $ and $ A\left( 2 \right) > A\left( 3 \right)  $  }
        \State \Return A(1)
    \ElsIf{$ A\left( 1 \right) < A\left( 2 \right) $ and $ A\left( 2 \right) > A\left( 3 \right)  $  }
        \State \Return A(2)
    \ElsIf{$ A\left( 1 \right) < A\left( 2 \right) $ and $ A\left( 2 \right) < A\left( 3 \right)  $  }
        \State \Return A(3)
    \EndIf
\EndIf

\If{$ m > 3 $ }
\State $ k \leftarrow \left\lfloor \frac{m}{2} \right\rfloor $
    \If{$ A\left( k-1  \right) > A\left( k \right) $ and $ A\left( k \right) > A\left( k+1 \right)  $  }
        \State \Return FindPeak($ A\left( 1:k \right)  $ )
    \ElsIf{$ >\left( k-1  \right) < A\left( k \right) $ and $ A\left( k+1 \right) > A\left( k \right)  $  }
        \State \Return FindPeak($ A\left( k:m \right)  $ )
    \ElsIf{$ >\left( k-1  \right) < A\left( k \right) $ and $ A\left( k+1 \right) < A\left( k \right)  $  }
        \State \Return A(k)
    \EndIf
\EndIf
\EndProcedure

\end{algorithmic}
\end{breakablealgorithm}

算法时间复杂度分析:
上述算法的时间复杂度 $ T\left( n \right)  $  满足下面的式子,设 $ n = 2^{k} $ 则 $ k = \log n$ 

\[
    \begin{aligned}
        T\left( n \right)  &= T\left( \frac{n}{2} \right) + 2 \\
        &= T\left( 2^{k-1} \right) + 2 \\
        &= T\left( 2^{k-2} \right) + 2 \times 2 \\
        &= \cdots \\
        &= T\left( 2^{0}\right)  + k \times 2\\
        &= T\left( 1 \right) + 2 \log n \\
        &= \Theta\left( \log n \right)  \\
    .\end{aligned}
\]

\end{document}
